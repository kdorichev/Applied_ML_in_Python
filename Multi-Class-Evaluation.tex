\section{Multi-Class Evaluation}
\begin{multicols}{2}

Now that we've looked at evaluation of binary classifiers, let's take a look at how the more general case of multi class classification is handled in evaluation. So in may respects, multi-class evaluation is a straightforward extension of the methods we use in binary evaluation. Instead of two classes, we have multiple classes. So, the results for multi-class evaluation amount to a collection of true verses predicted binary outcome per class. And just as we saw in the binary case, you can generate confusion matrices in the multi-class case. They're especially useful when you have multiple classes, because there are many different kinds of errors that result from one true class being predicted as a different class. We'll look at an example of that. Classification reports that we saw in the binary case are easy to generate for the multi-class case. Now the one area, which is worth a little more examination is how averaging across classes takes place. 

There are different ways to average multi-class results that we'll cover shortly. And the support, the number of instances for each class is important to consider. So just as we're all interested in how to handle imbalance classes in the binary case, it's important as you will see to consider similar issues of how the support for classes might vary to a large or small extent across multiple classes. There is a case of multi-label classification in which each instance could have multiple labels. For example, a web page might be labeled with different topics that come from a predefined set of areas of interest. We won't cover multi-label classification in this lecture. Instead, we'll focus exclusively on multi-class evaluation. The multi-class confusion matrix is a straightforward extension of the binary classifier two by two confusion matrix. 

For example, in our digits data set, there are ten classes for the digits, zero through nine. So, the ten class confusion matrix is a ten by ten matrix with the true digit class indexed by row and the predicted digit class indexed by column. 

As with the two by two case, the correct prediction is by the classifier where the true class matches the predicted class are all along the diagonal and misclassifications are off the diagonal. 

In this example which was created using the following notebook code 

based on a support vector classifier with linear kernel, we can see that most of the predictions are correct with only a few misclassifications here and there. 

The most frequent type of mistake here is apparently misclassifying the true digit, eight as a predicted digit one which happened three times. 

And indeed, the overall accuracy is high, about 97\% as shown here. As an aside, it's sometimes useful to display a confusion matrix as a heat map in order to highlight the relative frequencies of different types of errors. So, I've included the code to generate that here. For comparison, I've also included a second confusion matrix on the same dataset for another support vector classifier that does much worse in a distinctive way. The only change is to use an RBF, radial basis function kernel instead of a linear kernel. 

While we can see for the accuracy number were about 43\% below the confusion matrix that the classifier is doing much worse than the delinear kernel, that single number doesn't give much insight into why. 

Looking at the confusion matrix, however, reveals that for every true digit class, a significant fraction of outcomes are to predict the digit four. That's rather surprising. For example, of the 44 instances of the true digit 2 in row 2, 17 are classified correctly, but 27 are classified as the digit 4. Clearly, something is broken with this model and I picked this second example just to show an extreme example of what you might see when things go quite wrong. This digits dataset is well-established and free of problems. But especially when developing with a new dataset, seeing patterns like this in a confusion matrix could give you valuable clues about possible problems, say in the feature pre-processing for example. So as a general rule of thumb as part of model evaluation, I suggest always looking at the confusion matrix for your classifier. To get some insight into what kind of errors it is making for each class including whether some classes are much more prone to certain kinds of errors than others. 

Next, just as in the binary case, you can get a classification report that summarizes multiple evaluation metrics for a multi-class classifier with an average metric computed for each class. 

Now what I'm about to describe, also applies to the binary class case, but it's easier to see when looking at multi-class classification problem with several classes. 

So, here's an example of how to compute macro-average precision and micro-average precision on a sample dataset that I have extracted from our fruit dataset. In this example, we have three columns where the first column is the true class of an example. The second column is the predictive class from some classifier and the third column is a binary variable that denotes whether the predictive class matches the two class. 

And here, we have in our, this is a multi-class classification problem. 

And so, we have three classes here. We have several instances there, the orange class. We have two instances that are the lemon class and we have two instances that are the apple class. 

So in this first example, we'll compute macro-average precision and the key aspect of macro-average precision is that each class has equal weight. So in this case, each of these classes will contribute one-third weight towards the final macro-average precision value. 

So, there are two steps to compute macro-average precision. The first one is to compute the metric. So in this case, we're going to compute precision within each class. So, let's take a look at the orange class. There are five total examples in the orange class and only one of them was predicted correctly by the classifier. And so, that leads to a precision for the orange class of 1 out of 5 or 0.20. For the second class, the lemon class. There are a total of two instances and only one of them was predicted correctly, and that leads to a precision of one-half or 0.50 for the lemon class. Let's write the precision for each of the classes that we have calculated. 


This was 0.5. 

And for the third class, the apple class. The classifier predicted both of these correctly. So, that's a precision of 2 out of 2 or 1.0. 

That's the first step. We've computed the precision metric within each class. And then in the second step, we simply average across these three to produce the final result, to get our final macro-average precision. And so we can simply compute the average of 0.2, 0.5 and 1 and we get our final macro-average precision for this set of results of 0.57. You'll notice here that no matter how many instances they were in each class, because we computed position within each class first, each class contributes equally to the overall macro-average. So we could have had, for example, a million examples and from the orange class. But that class would have still been weighted equally, because we would have first computed precision for the million orange examples and then that number would still get a third of the weight compared to the other two classes. So, that's macro-average precision. 

Micro-average precision is computed a little differently and it gives each instance in the data results here equal weight. In micro-average precision, we don't compute precision for each class separately. We treat the entire dataset, the entire set of results here as an aggregate outcome. So to compute micro-average precision, we simply look at how many of all the examples. We have nine examples here in total and micro-average precision will simply compute the precision for all the examples, regardless of class in the set of results. So out of these nine instances, we have found that the classifier predicted four of them correctly. 

And so, the micro-average precision is simply computed as 4/9 or 0.44. And you'll notice here that if we had a million instances of the orange class, for example, that with micro-average precision, because each instance has equal weight. That would lead to the orange class contributing many, many more instances to our overall micro-average precision. And so, the effect of micro-average precision is to give classes with a lot more instances much more influence. So, the average here would have been influenced much more by the million orange examples than by the two lemon and apple examples. 

And so, that is the difference between micro and macro-average precision. If the classes have about the same number of instances, macro and micro-average will be about the same. If some classes are much larger, have more instances than others and you want to weight your metric toward the largest ones, use micro-averaging. If you want to weight your metric towards the smallest classes, use macro-averaging. If the micro-average is much lower than the macro-average, then examine the larger classes for poor metric performance. If the macro-average is much lower than the micro-average, then you should examine the smaller classes to see why they have poor metric performance. Here, we use the average parameter on the scoring function. In the first example, we used the precision metric and specify whether we want micro-average precision which is the first case or macro-average precision in the second case. 

In the second example, we use the F1 metric and compute micro and macro-averaged F1. 

Now that we've seen how to compute these metrics, let's take a look at how to use them to do model selection. 

\end{multicols}