\section{Linear Classifiers: Support Vector Machines}
\begin{multicols}{2}

Let's look at how linear models are also used for classification, starting with binary classification. 

This approach to prediction uses the same linear functional form as we saw for regression. But instead of predicting a continuous target value, we take the output of the linear function and apply the sine function to produce a binary output with two possible values, corresponding to the two possible class labels. 

If the target value is greater than zero, the function returns plus one and if it's less than zero, the function returns minus one. Here's a specific example. 

So let's take a look at a simple linear classification problem, a binary problem that has two different classes, and where each data instance is represented by two informative features. So we're going to call these features x1 on the x-axis and x2 on the y-axis. Now, how do we get from taking a data instance that's described by a particular combination of x1, x2 to a class prediction. 

Well in a linear classifier, what we do is we take the feature values assigned to a particular data point. So in this case, it might be 2.1,0. 

So that represents the x1 and x2 feature values for this particular instance, let's say, and so that corresponds to the input here. And then we take those two x1, x2 values and put them through a linear function f here. And the output of this f needs to be a class value. So either we want it to return +1, if it's in one class and -1 if it's in the other class. So the linear classifier does that by computing a linear function of x1, x2 that's represented by this part of the equation here. 

So this w is a vector of weights. This x represents the vector of feature values. And then b is a bias term that gets added in. So this is a dot product, this circle here is a dot product. Which means that, let's say, the simple case where we had (w1, w2), and a future vector of (x1, x2). The dot product of those is simply the linear combination of w1 x1 + w2 x2. 

We take the X values, and we have certain weights that are learned for the classifier. So we compute w1 x1 + w2 x2. Then we feed that, and then plus a bias term, if there is one. 

And we feed the output of that through the sign function that converts the value in here. If it's above 0, it'll convert to +1, and if it's below 0, it'll convert it to -1 and that's what the equation represents here. So that is the very simple rule that we use to convert a data instance with its input features to an output prediction. Okay, let's take a look using a specific linear function to see how this works in practice. So what I've done here is drawn a straight line through the space that is defined by the equation x1- x2 = 0. In other words, every point on this line satisfies this equation. So you can see, for example, that x1 = -1 and x2 = -1 and if you subtract them you get 0. Essentially, it's just the line that represents all the points where x1 and x2 are equal. 

So this corresponds to, so we can rewrite this x1- x2 = 0 into a form that uses a dot product of weights with the input vector x. So in order to get x1- x2 = 0 that's equivalent writing a linear function where you have a weight vector of 1 and -1 and a bias term of 0 So for example, so if we have weights (w1, w2) dot (x1, x2). We call that this is just the same as computing w1 x1 + w2 x2. And so in this case, if w is 1 and -1, that's equivalent to x1- x2. 

So all I've done here then is just convert the description of this line into a form that can be used as a decision rule for classifier. So I might have to look at a specific point. So suppose that we wanted to classify the point here that had coordinates -0.75 and -2.25. So this point right here. So all we would do to have a classifier make a decision with this decision boundary would be to plug in those coordinates into this part of the function, apply the weights and the bias term that describe the decision boundary. 

So here we're computing. So this expression here corresponds to this part of the equation. 

And so, if we compute w1 times x1 + w2 times x2 plus 0, we get a value of 0.15 that's inside the sign function. And then the sign function will output 1 if this value is greater than zero, which it is or minus 1, if the value is less than zero. So in this case, because it's greater than zero, the output from the decision function would be plus 1. So this has classified this point as being class one. 

If we look at a different point, let's take a look at the point -1.75 and -0.25. So again, we're just going to up, so this is corresponding to the value of x1 and x2. And again, if we just take these values and plug them into this part of the equation. Apply the weights and the bias term that describe the decision boundary as you did before. We do this computation, we find out that, in fact, classifier predicts a class of -1 here. So you see that by applying a simple linear formula, we've been able to produce a class value for any point in this two dimensional features space. 

So one way to define a good classifier is to reward classifiers for the amount of separation that can provide between the two classes. And to do this, we need define the concept of classifier margin. So informally, for our given classifier, The margin is the width that the decision boundary can be increased before hitting a data point. So what we do is we take the decision boundary, and we grow a region around it. Sort of in this perpendicular to the line in this direction and that direction, and we grow this width until we hit a data point. So in this case, we were only able to grow the margin a small amount here, before hitting this data point. So this width here between the decision boundary and nearest data point represents the margin of this particular classifier. 

Now you can imagine that for every classifier that we tried, we can do the same calculation or simulation to find the margin. 

And so among all possible classifiers that separate these two classes then, we can define the best classifier as the classifier that has the maximum amount of margin which corresponds to the one shown here. So you recall that the original classifier on the previous slide had a very small margin. This one manages to achieve a must larger margin. So again, the margin is the distance the width that we can go from the decision boundary perpendicular to the nearest data point. 

So you can see that by defining this concept of margin that sort of quantifies the degree to which the classifier can split the classes into two regions that have some amount of separation between them. We can actually do a search for the classifier that has the maximum margin. This maximum margin classifier is called the Linear Support Vector Machine, also known as an LSVM or a support vector machine with linear kernel. Now we'll explain more about what the concept of a kernel is and how you can define nonlinear kernels as well as kernels, and why you'd want to do that. We'll cover those shortly in a continuation of our support vector machine lecture later in this particular week. Here's an example in the notebook on how to use the default linear support vector classifier in scikit-learn, which is defined in the sklearn SVM library. 

The linear SVC class implements a linear support vector classifier and is trained in the same way as other classifiers, namely by using the fit method on the training data. Now in the simple classification problem I just showed you, the two classes were perfectly separable with a linear classifier. In practice though, we typically have noise or just more complexity in the data set that makes a perfect linear separation impossible, but where most points can be separated without errors by linear classifier. And our simple binary classification dataset here is an illustration of that. So how tolerant the support vector machine is of misclassifying training points, as compared to its objective of minimizing the margin between classes is controlled by a regularization parameter called C which by default is set to 1.0 as we have here. 

Larger values of C represent less regularization and will cause the model to fit the training set with these few errors as possible, even if it means using a small immersion decision boundary. Very small values of C on the other hand use more regularization that encourages the classifier to find a large marge on decision boundary, even if that decision boundary leads to more points being misclassified. Here's an example in the notebook showing the effect of varying C on this basic classification problem. On the right, when C is large, the decision boundary is adjusted so that more of the black training points are correctly classified. While on the left, for small values of C, the classifier is more tolerant of these errors in favor of capturing the majority of data points correctly with a larger margin. 

Finally, here's an example of applying the Linear Support Vector Machine to a real world data set, the breast cancer classification problem. And here, we can see that it achieves reasonable accuracy without much parameter tuning. 

On the positive side, linear models, in the case of linear and logistic regression, are simple and easy to train. And for all types of linear models, prediction's very fast because, of the linear nature of the prediction function. 

Linear models including Linear Support Vector Machines also perform effectively on high dementional data set, especially, in cases where the data instances are sparse. Linear Models scale well to very large datasets as well. 

In the case of Linear Support Vector Machines, they only use a subset of training points and decision function. These training points are called support vectors. 

So the algorithm can be implemented in a memory efficient way. 

\end{multicols}