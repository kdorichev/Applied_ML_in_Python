\section{Kernelized Support Vector Machines}
\begin{multicols}{2}

We saw earlier how linear support vector machines served as effective classifiers for some datasets, by finding a decision boundary with maximum margin between the classes. 

Linear support vector machines worked well for simpler kinds of classification problems, where the classes were linearly separable or close to linearly separable like this example on the left. 

But with real data, many classification problems aren't this easy. With the different classes located in future space in a way that a line or hyperplane can't act as an effective classifier. Here's an example on the right. 

These dataset is difficult, or impossible for a linear model, a line or hyperplane, to classify well. 

So to help address the situation, we're now going to turn to our next type of supervised learning model. A very powerful extension of linear support vector machines called kernelized support vector machines. 

Basically, kernelized support vector machines, which I'll just call SVMs, can provide more complex models that can go beyond linear decision boundaries. 

As with other supervised machine learning methods, SVMs can be used for both classification and regression. 

But due to time constraints, we'll focus only on classification in this lecture. 

We also won't go into the mathematical details of how SVMs operate, but we will give an high level overview that hopefully captures the most important elements of this method. 

In essence, one way to think about what kernelized SVMs do, is they take the original input data space and transform it to a new higher dimensional feature space, where it becomes much easier to classify the transform to data using a linear classifier. 

Here's a simple 1-dimensional example of what I mean. 

Here's a binary classification problem in 1-dimension with a set of points that lie along the x-axis. Color black for one class and white for the second class. 

Each data point here has just one feature. It's positioned on the x-axis. If we gave this problem to a linear support vector machine, it would have no problem finding the decision boundary. It gives the maximum margin between points of different classes. 

Here I've engineered the data points, so that the maximum margin decision boundary happens to be at x = 0. Now suppose we gave the linear support vector machine a harder problem, where the classes are no longer linearly separable. 

A simple linear decision boundary just doesn't have enough expressive power to classify all these points correctly. 

So what can we do about this? 

One very powerful idea is to transform the input data from a 1-dimensional space to a 2-dimensional space. 

We can do this, for example, by mapping each 1-dimensional input data instance xi 

to a corresponding 2-dimensional ordered pair xi, xi squared, whose new second feature is the squared value of the first feature. 

We're not adding in any new information in the sense that all we need to obtain this new 2-dimensional version is already present in the original 1-dimensional data point. 

This might remind you of a similar technique that we saw when adding polynomial features to a linear regression problem earlier in the course. 

We can now learn a linear support vector machine in this new, 2-deminsional feature space, whose maximum margin decision boundary might look like this here to correctly classify the points. 

Any future 1-dimensional points for which we'd like to predict the class, we can just create the 2-deminsional transformed version and predict the class of the 2-deminsional point, using this 2-deminsional linear SVM. If we took the inverse of the transformation we just applied to bring the data points back to our original input space, 

we can see that the linear decision boundary in the 2-deminsional space corresponds to the two points where a parabola crosses the x-axis. Now just so that this very important idea is clear, let's move from a 1-dimensional problem to a 2-deminsional classification problem. You can see the same powerful idea in action here. 

Here we have two classes represented by the black and the white points. Each of which has two features, x0 and x1. 

The points of both classes are scattered around the origin 00 in a 2-deminsional plane. The white points form a cluster right around 00, that's completely surrounded by the black points. 

Again, this looks to be impossible for a linear classifier, which in 2-deminsional space is a straight line, to separate the white points from the black points with any degree of accuracy. 

But just as we did in the 1-dimensional case, we can map each 2-deminsional point (x0,x1) to a new 3-deminsional point by adding a third feature. Mathematically 1-(x0 squared+x1 squared), and this transformation acts to shape the points into a parabaloid around (0,0). Now the wide points since they're close to (0,0), get mapped to points with higher vertical z values, that new third feature that are close to 1. While the black points which are farther from (0,0) get mapped to points with z values that either close to 0 or even negative. 

With this transformation, it makes it possible to find a hyperplane. Say, z = 0.9, that now easily separates the white data points that are near z = 1 from most or all of the black data points. Finally, the decision boundary consists of the set of points in 3-deminsional space where the paraboloid intersects the maximum margin hyperplane decision boundary. 

This corresponds to an ellipse-like decision boundary in 2-deminsional space that separates the white points from the black points in the original input space. 

This idea of transforming the input data points to a new feature space where a linear classifier can be easily applied, is a very general and powerful one. 

There are lots of different possible transformations we could apply to data. And the different kernels available for the kernelized SVM correspond to different transformations. Here we're going to focus mainly on what's called the radial basis function kernel, which we'll abbreviate as RBF. 

And also look briefly look at something called a polynomial kernel, that's also included with scikit-learns SVM module. The kernel function in an SVM tells us, given two points in the original input space, what is their similarity in the new feature space? 

For the radial basis function kernel, the similarity between two points and the transformed feature space is an exponentially decaying function of the distance between the vectors and the original input space as shown by the formula here. 

So what does the radial basis function feature transformation look like? 

Well, this diagram should give an intuitive idea. Here's another graphical illustration of a binary classification problem. 

And again this diagram is for illustration purposes only and it's just an approximation, but essentially you can think of it in a similar way to the 2-deminsional to 3-deminsional example we saw earlier. 

So on the left, is a set of samples in the input space. The circles represent the training points of one class and the squares represent training points of a second class. 

On the right, the same samples are shown in the transformed feature space. 

Using the radial basis function kernel in effect, transforms all the points inside a certain distance of the circle class to one area of the transformed feature space. And all the points in the square class outside a certain radius get moved to a different area of the feature space. 

The dark circles and squares represents the points that might lie along the maximum margin for a support vector machine in the transformed feature space. 

And also, it shows the corresponding points in the original input space. So just as we saw with the simple 1D and 2D examples earlier, the kernelized support vector machine tries to find the decision boundary with maximum margin between classes using a linear classifier in the transformed feature space not the original input space. 

The linear decision boundary learn feature space by linear SVM corresponds to a non-linear decision boundary In the original input space. So in this example, an ellipse like closed region in the input space. Now, one of the mathematically remarkable things about kernelized support vector machines, something referred to as the kernel trick, is that internally, the algorithm doesn't have to perform this actual transformation on the data points to the new high dimensional feature space. 

Instead, the kernelized SVM can compute these more complex decision boundaries just in terms of similarity calculations between pairs of points in the high dimensional space where the transformed feature representation is implicit. 

This similarity function which mathematically is a kind of dot product is the kernel in kernelized SVM. 

And for certain kinds of high dimensional spaces, the similarity calculation between points, the kernel function can have a simple form like we see with the radial bases function calculation. This makes it practical to apply support vector machines, when the underlying transformed feature space is complex or even infinite dimensional. 

Even better, we could easily plug in a variety of different kernels choosing one to suit the properties of our data. 

Again, different choices of kernel correspond to different types of transformations to that higher dimensional feature space. 

Here's the result of using a support vector machine with RBF kernel, on that more complex binary classification problem we saw earlier. 

You can see that unlike a linear classifier, the SVM with RBF kernel finds a more complex and very effective set of decision boundaries that are very good at separating one class from the other. 

Note that the SVM classifier is still using a maximum margin principle to find these decision boundaries. But because of the non-linear transformation of the data, these boundaries may no longer always be equally distant from the margin edge points in the original input space. 

Now let's look at an example of how we did this using scikit-learn in Python. 

To use SVMs, we simply import the SVC class from sklearn.svm and use it just as we would in the other classifier. For example, by calling the fit method with the training data to train the model. 

There is an SVC parameter called kernel, that allows us to set the kernel function used by the SVM. 

By default, the SVM will use the radial base's function, but a number of other choices are supported. 

Here in the second example and plot, we show the use of the polynomial kernel instead of the RBF kernel. 

The polynomial kernel, using the kernel poly setting, essentially represents a future transformation similar to the earlier quadratic example. In the lecture, this future space represented in terms of futures that are polynomial combinations of the original input features, much as we saw also for linear regression. 

The polynomial kernel takes additional parameter degree that controls the model complexity and the computational cost of this transformation. 

You may have noticed that the RBF kernel has a parameter gamma. 

Gamma controls how far the influence of a single trending example reaches, which in turn affects how tightly the decision boundaries end up surrounding points in the input space. 

Small gamma means a larger similarity radius. So that points farther apart are considered similar. Which results in more points being group together and smoother decision boundaries. 

On the other hand for larger values of gamma, the kernel value to K is more quickly and points have to be very close to be considered similar. This results in more complex, tightly constrained decision boundaries. 

You can see the effect of increasing gamma that is sharpening the kernel in this example from the notebook. 

Small values of gamma give broader, smoother decision regions. While larger values of gamma give smaller, more complex decision regions. 

You can set the gamma parameter when creating the SVC object to control the kernel width in this way, as shown in this code. 

You may recall from linear SVMs that SVMs also have a regularization parameter, C, that controls the tradeoff between satisfying the maximum margin criterion to find the simple decision boundary, and avoiding misclassification errors on the training set. The C parameter is also an important one for kernelized SVMs, and it interacts with the gamma parameter. 

This example from the notebook shows the effect of varying C and gamma together. 

If gamma is large, then C will have little to no effect. Well, if gamma is small, the model is much more constrained and the effective C will be similar to how it would affect a linear classifier. 

Typically, gamma and C are tuned together, with the optimal combination typically in an intermediate range of values. For example, gamma between 0.0001 and 10 and see between 0.1 and 100. Though the specifical optimal values will depend on your application. 

Kernelized SVMs are pretty sensitive to settings of gamma. The most important thing to remember when applying SVMs is that it's important to normalize the input data, so that all the features have comparable units that are on the same scale. We saw this earlier with some other learning methods like regularized regression. 

Let's apply a support vector machine with RBF kernel to a real world dataset to see why this normalization is important. 

Here, we'll apply a support vector machine with RBF kernel to the breast cancer dataset. 

Note that we're not touching the input data in anyway, we're simply passing in the raw values. 

We can see the results with training set accuracy of 1.00 and the test set accuracy of 0.63 that show that the support vector machine is over fitting. It's doing well on the training data, but very poorly on the test data. 

Now let's add a MinMaxScaler transformation of the training data. Remembering to also apply the same scaler to the test data. 

After this scaler has been applied, all the input features now lie in the same range zero to one. 

And looking at these new results, the tests set accuracy is much, much higher, 96%. This illustrates what a huge difference normalizing the features of the training data can have on SVM performance. 

Let's review the strengths and weaknesses of support vector machines. 

On the positive side, support vector machines perform well on a range of datasets, and have been successfully applied on data that ranges from text to images and many more types. 

The support vector machine's also potentially very versatile, due to its ability to specify different kernel functions, including possible custom kernel functions depending on the data. 

Support vector machines also typically work well for both low and high-dimensional data. Including data with hundreds, thousands or even millions of sparse dimensions. This makes it well suited to test classification for example. On the negative side as the training set size increases, the run time, speed, and memory usage in the SVM training phase also increases. So for a large datasets with hundreds of thousands, or millions of instances, an SVM may become less practical. 

As we saw when applying a support vector machine to a real world dataset, using an SVM requires careful normalization of the input data as well as parameter tuning. 

The input should be normalized that all features have comparable units and around similar scales if they aren't already. 

Support vector machines also don't provide direct probability estimates for predictions, which are needed for some applications. Now, there are ways to estimate these probabilities using techniques such as plot scaling, which transforms the output of the classifier to a probability distribution over classes by fitting a logistic regression model to the classifiers course. Finally, it could be difficult to interpret the internal model parameters of a support vector machine. Which means the applicability of support vector machines in scenarios where interpretation is important for people may be limited when we want to understand why a particular prediction was made. 

As a reminder, there are three main parameters that control model complexity for kernelized SVMs. 

First, there's the kernel type which defaults to RBF for radial basis function. But several other common types are available in scikit-learns SVC module. 

Second, each kernel has one or more kernel specific parameters that control aspects like the influence of training points according to their distance. 

In the case of the RBF kernel, SVM performance is very sensitive to the setting of the gamma parameter that controls the kernel width. 

Finally, for any support vector machine, the C: regularization parameter operates as we've discussed before. And it's typically tuned with the kernel parameters such as gamma for optimal performance. 

\end{multicols}